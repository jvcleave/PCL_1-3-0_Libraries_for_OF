%---------------------------Skew---------------------------
\section{Skew\label{s:hex-skew}}

To compute the skew, we'll need to compute normalized versions of the principal axes:
\[
\begin{array}{lcl}
\hat{X_1} &=& \frac{\vec X_1}{\normvec{ X_1 }}\\
\hat{X_2} &=& \frac{\vec X_2}{\normvec{ X_2 }}\\
\hat{X_3} &=& \frac{\vec X_3}{\normvec{ X_3 }}
\end{array}
\]
Skew measures the degree to which a pair of vectors are parallel using the dot product.
This means we have three skews to consider for a hexahedron, each of which is the
absolute value of the cosine of the angle between two principal axes:
\[
\begin{array}{lcl}
skew_{12} &=& \left| \hat{X_1} \cdot \hat{X_2} \right|\\
skew_{13} &=& \left| \hat{X_1} \cdot \hat{X_3} \right|\\
skew_{23} &=& \left| \hat{X_2} \cdot \hat{X_3} \right|.
\end{array}
\]
The metric is then the maximum of these skews
\[
  q = \max\left\{ skew_{12}, skew_{13}, skew_{23} \right\}
\]

Note that if $\normvec{X_1}$ or $\normvec{X_2}$ or $\normvec{X_3} \leq DBL\_MIN$, we set $q = DBL\_MAX$.

\hexmetrictable{skew}%
{$1$}%                                        Dimension
{$[0,0.5]$}%                                  Acceptable range
{$[0,1]$}%                                    Normal range
{$[0,DBL\_MAX]$}%                             Full range
{$0$}%                                        Cube
{Adapted from \cite{tf:89}}%                  Citation
{v\_hex\_skew}%                               Verdict function name
