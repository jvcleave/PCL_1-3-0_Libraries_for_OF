%---------------------------Condition-----------------------------
\section{Condition}

First, define
\[
\begin{array}{lcl}
\vec C_1 &=& \vec L_0 \\
\vec C_2 &=& \left(-2 \vec L_2 - \vec L_0\right)/\sqrt{3} \\
\vec C_3 &=& \left(3 \vec L_3 + \vec L_2 - \vec L_0\right)/\sqrt{6}
\end{array}
\]
\[
C_{det} = \vec C_1 \cdot ( \vec C_2 \times \vec C_3 ),
\]

and
\[
\begin{array}{lcl}
T_1 &=& \vec C_1 \cdot \vec C_1 + \vec C_2 \cdot \vec C_2 + \vec C_3 \cdot \vec C_3\\
T_2 &=& (\vec C_1 \times \vec C_2 ) \cdot (\vec C_1 \times \vec C_2) + 
      (\vec C_2 \times \vec C_3 ) \cdot (\vec C_2 \times \vec C_3) + 
      (\vec C_1 \times \vec C_3 ) \cdot (\vec C_1 \times \vec C_3)
\end{array}
\]

The condition metric is then defined as follows:
\begin{equation*}
q = \frac{ \sqrt{ T_1 T_2 } } { 3 C_{det} }.
\end{equation*}

Note that if If $C_{det} \leq DBL\_MIN$, we set $q = DBL\_MAX$.

\tetmetrictable{condition}%
{$1$}%                  Dimension
{$[1,3]$}%              Acceptable range
{$[1,DBL\_MAX]$}%       Normal range
{$[1,DBL\_MAX]$}%       Full range
{$1$}%                  Equilateral tet
{\cite{knu:00}}%        Citation
{v\_tet\_condition}%                            Verdict function name


